% MA211 - Lecture 23 (THE END!)
\documentclass[pdftex, xcolor=pdftex, dvipsnames,handout]{beamer}

\usetheme{MA211}
\usepackage{thumbpdf}
\usepackage{wasysym}
%\usepackage{ucs}
\usepackage[utf8]{inputenc}
\usepackage{pgf,pgfarrows,pgfnodes,pgfautomata,pgfheaps,pgfshade}
\usepackage{verbatim}

\usepackage{eurosym}
\usepackage{euler}

\usepackage{calc}               % Simple computations with LaTeX variables
%\usepackage[hang]{caption2}     % Improved captions

\usepackage{graphicx}           % Standard graphics package

\usepackage{amsmath, amsthm, amssymb}


\newcommand{\fquad}{\mbox{\qquad}}
\newcommand{\bull}{$\bullet$ }

\newcommand {\I} {\mathcal I}
\newcommand {\calI} {\mathcal I}
\def\disint{\displaystyle\int}

\DeclareMathOperator{\D}{d}
\newcommand{\dydx}{\frac{\D y}{\D x}}

%\definecolor{gray}{rgb}{0.69, 0.69, 0.69} \newcommand{\gray}[1]{\textcolor{gray}{#1}}
\definecolor{dogreen}{rgb}{0.33, 0.42, 0.18} \newcommand{\dogreen}[1]{\textcolor{dogreen}{#1}}
\definecolor{maroon}{rgb}{.5,0.2,0.2}\newcommand{\maroon}[1]{\textcolor{maroon}{#1}}
\definecolor{greena}{rgb}{.1,0.581,0.1}\newcommand{\greena}[1]{\textcolor{greena}{#1}}

\definecolor{blue4}{rgb}{0,0,.545}
\newcommand{\Blue}[1]{\textcolor{blue}{#1}}
\newcommand{\Red}[1]{\textcolor{red}{#1}}
\definecolor{pink}{rgb}{1.,0.75,0.8}
\definecolor{darkred}{rgb}{0.5,0.0,0.0}
\definecolor{darkgreen}{rgb}{0,0.3,0.3}
\definecolor{purple}{rgb}{0,0.3,0.3}
\definecolor{darkblue}{rgb}{0.0, 0.0, .5}
\definecolor{dpurple}{rgb}{.3,.0,.3}
\newcommand{\Green}[1]{\textcolor{darkgreen}{#1}}
\newcommand{\DRed}[1]{\textcolor{darkred}{#1}}
\newcommand{\DBlue}[1]{\textcolor{darkblue}{#1}}
\newcommand{\Purple}[1]{\textcolor{dpurple}{#1}}
\newcommand{\Emph}[1]{\textcolor{darkred}{\textbf{\it #1}}}
\newcommand{\remph}[1]{\textcolor{darkred}{\textbf{\emph{#1}}}}
\newcommand{\bemph}[1]{\textcolor{darkblue}{\textbf{\emph{#1}}}}
\newcommand{\gemph}[1]{\textcolor{darkgreen}{\textbf{\emph{#1}}}}
\newcommand{\Bf}[1]{\textcolor{darkblue}{\textbf{#1}}}
\newcommand{\Gf}[1]{\textcolor{darkgreen}{\textbf{#1}}}
\newcommand{\Rf}[1]{\textcolor{red}{\textbf{#1}}}
\newcommand{\Rmf}[1]{\textcolor{red}{\mathbf{#1}}}

\newcommand{\Conj}[1]{\overline{#1}}

\newcommand{\code}[1]{\textcolor{darkblue}{\texttt{\textbf{#1}}}}
\newcommand{\icode}[1]{{\blue\texttt{\textbf{\emph{#1}}}}}
\newcommand{\gcode}[1]{{\Green{\texttt{\textbf{\emph{#1}}}}}}
\newcommand{\out}[1]{\texttt{\emph{\textbf{\Green{#1}}}}}





\newenvironment{vminipage}%
{\begin{Sbox}\begin{minipage}\begin{small}\begin{verbatim}}%
{\end{verbatim}\end{small}\end{minipage}\end{Sbox}\fbox{\TheSbox}}

\newenvironment{nminipage}%
{\begin{Sbox}\begin{minipage}}%
{\end{minipage}\end{Sbox}\fbox{\TheSbox}}


\let\Arg\relax\DeclareMathOperator{\Arg}{\mathtt{Arg}}
\let\Arg\relax\DeclareMathOperator{\e}{\mathtt{e}}

\newcommand {\AND} {\wedge}
\newcommand {\OR} {\vee}
\newcommand {\NOT} {\neg}
\newcommand {\IMPLIES} {\rightarrow}
%\newcommand {\IFF} {\leftrightarrow}
\renewcommand {\iff} {\Leftrightarrow}
\newcommand {\NAND} {\uparrow}
\newcommand {\NOR} {\downarrow}
\newcommand {\XOR} {\otimes}

\newenvironment{citemize}% Colour items
{\begin{description}}%
{\end{description}}

\newcommand {\maroonitem}{\item[\maroon{$\bullet$}]}

\newcommand {\gitem} {\item {\includegraphics[width=.4cm,angle=-10]{img/green-bullet-on-white.ps}}}
\newcommand {\ritem} {\item {\includegraphics[width=.4cm,angle=-10]{img/red-bullet-on-white.ps}}}
\newcommand {\yitem} {\item {\includegraphics[width=.4cm,angle=-10]{img/yellow-bullet-on-white.ps}}}
\newcommand {\bitem} {\item {\includegraphics[width=.4cm,angle=-10]{img/blue-bullet-on-white.ps}}}

\newcommand {\greenitem} {\item {\includegraphics[width=.4cm,angle=-10]{img/green-bullet-on-white.ps}}}
\newcommand {\reditem} {\item {\includegraphics[width=.4cm,angle=-10]{img/red-bullet-on-white.ps}}}
\newcommand {\yellowitem} {\item {\includegraphics[width=.4cm,angle=-10]{img/yellow-bullet-on-white.ps}}}
\newcommand {\blueitem} {\item {\includegraphics[width=.4cm,angle=-10]{img/blue-bullet-on-white.ps}}}

\newcommand {\eq}[1]%
  {$\DBlue{#1}$}
\newcommand {\eqd}[1]%
  {$\displaystyle\DBlue{#1}$}
%\newcommand{\eq}[1]{\boldmath \DBlue{$#1$}}


\newcommand {\csf}{\centerslidesfalse}
\newcommand {\cst}{\centerslidestrue}

\newcommand {\vecii}[2] {   \big(\begin{smallmatrix} #1 \\ #2 \end{smallmatrix}\big)}
\newcommand{\atwo}[2]{\left(\!\!\begin{array}{c} #1 \\ #2 \end{array}\!\!\right)}


\newcommand{\C}{\mathbb{C}}
\newcommand{\Q}{\mathbb{Q}}
\newcommand{\R}{\mathbb{R}}
\newcommand{\N}{\mathbb{N}}
\newcommand{\Z}{\protect\mathbb{Z}}  % protect for index.
\newcommand {\Rs}{ \mathbb{R}}
\newcommand {\Cs}{ \mathbb{C}}
\newcommand {\Rnn}{ \mathbb{R}^{n \times n}}
\newcommand {\Rn}{ \mathbb{R}^{n}}


\newcommand{\mblock}{%
\setbeamercolor*{block title}{bg=maroon,fg=white}
\setbeamercolor*{block body}{bg=white,fg=maroon}
}%

\newcommand{\bblock}{%
\setbeamercolor*{block title}{bg=Steel,fg=white}
\setbeamercolor*{block body}{bg=Mylightgray,fg=Steel}
}%

\newcommand{\gblock}{%
\setbeamercolor*{block title}{bg=Green,fg=white}
\setbeamercolor*{block body}{bg=Mylightgray,fg=darkgreen}
}%


\newcommand{\rblock}{%
\setbeamercolor*{block title}{bg=Red,fg=white}
\setbeamercolor*{block body}{bg=white,fg=Black}
}%

\def\disfrac{\displaystyle\frac}
\newcommand{\TakeNotes}{\includegraphics[width=2cm]{TakeNote}}

\def\eps{\varepsilon}
\newcommand {\del}[2]{ {\frac{\partial #1}{\partial #2}}}
\newcommand {\x}[1]{x^{[#1]}}
\newcommand {\delx}{ {\frac{\partial}{\partial x}}}
\newcommand {\delt}{ {\frac{\partial}{\partial t}}}
\newcommand {\dely}{ {\frac{\partial}{\partial y}}}
\newcommand {\ith}{{(i)}}
\renewcommand {\vec}[1]{ {\boldsymbol{#1}}}
\newcommand {\Oh} {\mathcal O}
\newcommand {\Err} {\mathcal E}
%\newcommand {\th} {\mathrm{th}}
\DeclareMathOperator{\fl}{fl}
\DeclareMathOperator{\sign}{sign}
\DeclareMathOperator{\Cond}{Cond} 
\DeclareMathOperator{\cond}{cond}
\DeclareMathOperator{\diag}{diag} 
\DeclareMathOperator{\sym}{sym} 
\DeclareMathOperator{\Trace}{Trace}

\DeclareMathOperator{\E}{e}

\newcommand {\Rsym}{{ \mathbb{R}^{n \times n}_\mathrm{sym}}}

\newcommand {\st} {\mathrm{st}}
\newcommand {\nd} {\mathrm{nd}}


\parskip .25cm


\theoremstyle{definition}
\newtheorem{exercise}{Exercise}[section]
\newtheorem{method}{Method}[section]

\newcommand{\Header}[1]{\begin{center}{\Large \Bf{#1}}\end{center}}

\subtitle{MA211}
\title{Lecture 23:\\ Integrating factors and course review}

\author{Dr Niall Madden}

\date{\Large Monday $24^\mathrm{th}$ Nov 2008}


\begin{document}

\setcounter{framenumber}{-1}
\frame{

\begin{columns}[c]
\column{0.45\textwidth}
\centering
\includegraphics[width=4cm]{images/EndIsNear}


\column{0.55\textwidth}
\begin{block}{}
\begin{center}
\begin{large}
 \insertsubtitle
\end{large}

\vspace{.1cm}

\begin{Large}
\textbf{\inserttitle}
\end{Large}


\vspace{.3cm}

{\insertdate}

\end{center}
\end{block}

\end{columns}

}

%\frame{
%  \frametitle{Topics of the day...}

%\begin{columns}[c]
%\column{0.5\textwidth}
% \tableofcontents
%\column{0.5\textwidth}
%See also Sections 9.3 and 9.5 of Stewart.

%\end{columns}
%}



\section{Integrating Factors}

\frame{

\Header{Summary of Technique of Integrating Factors}

Given a problem of the form:
\[
\frac{dy}{dx} + P(x)y = Q(x).
\]
\begin{enumerate}
\item Let  \eqd{v  = e^{\int P(x) dx}}.
\item Solve \eqd{(vy)' = v Q(x)} by integrating:
\[
vy = \int v Q(x) dx.
\]
not forgetting the constant of integration.

\item Divide by \eq{v} to get the solution:
\[
y = \frac{\int v Q(x) dx}{v}.
\]
\end{enumerate}
}

\subsection{Examples}
\frame{

\begin{example}
Solve the equation 
\[
x \disfrac{dy}{dx}+ y = \sin(x)
\]
subject to the initial condition $y(\pi /2) = 1$.
\end{example}

%\n\und{Solution}: $\disfrac{dy}{d\theta}+\disfrac{1}{\theta}y =
%\disfrac{1}{\theta}\sin\theta$. \\
%Integrating factor $v = e^{\int\frac{1}{\theta}\,d\theta} =
%e^{\ln\theta}=\theta$. 
%Then 
%\begin{eqnarray*}
%\theta\frac{dy}{d\theta} + y =\sin\theta & \Longrightarrow &
%\frac{d}{d\theta}(\theta y) = \sin\theta \\
%& \Longrightarrow & \theta y = -\cos\theta + C \\
%& \Longrightarrow & y(\theta) = \frac{-\cos\theta+C}{\theta} \\
%\end{eqnarray*}
%Now $y(\pi /2) = 1\Longrightarrow \disfrac{-0+C}{\pi
%/2}=1\Longrightarrow C=2/\pi$. Thus 
%$$
%y(\theta ) = -\frac{\cos\theta }{\theta}+\frac{\pi}{2\theta}.
%$$

\vspace{4cm}

}

\frame{

\begin{example}
Solve the equation 
$$
x\frac{dy}{dx}-y = x^3,\ \ y(1)=1.
$$
\end{example}
%\n\und{Solution}: $\disfrac{dy}{dx}+\left(-\disfrac{1}{x}\right)y =
%x^2$. \\
%Integrating factor $v(x) = e^{\int -1/x\,dx} = e^{-\ln x} =
%\disfrac{1}{x}$.
%Then 
%\begin{eqnarray*}
%\frac{1}{x}\frac{dy}{dx}-\frac{1}{x^2}y = x^2 & \Longrightarrow & 
%\frac{d}{dx}\left(\frac{y}{x}\right) = x^2 \\
%& \Longrightarrow & 
%\frac{y}{x} = \frac{x^3}{3}+C \\
%& \Longrightarrow & 
%y = \frac{x^4}{3}+Cx. 
%\end{eqnarray*}
%Now $y(1)=1\Longrightarrow \disfrac{1}{3}+C = 1,\ C =
%\disfrac{2}{3}$. \\
%Solution $y = \disfrac{x^4}{3}+\disfrac{2x}{3}$. 

\vspace{4cm}
}



\frame{
\begin{example}[Q3(c), Semester 1, '06/'07]
\[ e^x\frac{dy}{dx} + 2e^x y = 1.\]
\end{example}

\vspace{5cm}
}


\frame{
\begin{example}
Solve the following differential equation:
\[ 
\frac{dy}{dx} + \cos(x) y = 2xe^{-\sin(x)}.
\]
\end{example}

\vspace{5cm}
}


\frame{
\begin{exercise}
Solve the following differential equations:
\begin{enumerate}[(i)]
\item \eqd{y' + \frac{y}{x} = x^2 - \frac{1}{x}, \qquad y(1)=1/4.}
\item \eqd{y' + 2y = e^{-x}.}
\item \eqd{y' = x^2 + x^2y}
\item \eqd{y' + 3xy =x}
\item \eqd{y' = \sin(x)y = 3\sin(2x)}
\item \eqd{xy' + y = 2x \sin(x)}
\item \eqd{2xyy' =x^2 +3y^2}
\item \eqd{y' + \frac{y}{\tan(x)} = 3x+1}

\end{enumerate}
See also: Problem Set 5.
\end{exercise}
}

\section{Course Review}

\frame{
Over the past 12 weeks or so, we have covered the following topics


\begin{center}
\Bf{\Large Part 1: Functions, derivatives and integrals}
\end{center}

\begin{itemize}
\item Functions, including the ideas domain, codomain and range,
  one-to-one and onto, and the   inverse of a function. Even and off
  functions.

\item Limits, e.g., squeeze theorem, and l'Hopital's rule,

\item Derivatives, including differentiating the product and ratio of
  twp functions. The chain rule. Derivatives of trigonometric and
  inverse trig functions,

\item Antiderivatives and integrals; logs and exponentials. 

\item Euler's formula, and Hyperbolic functions.

\end{itemize}
}


\frame{
\begin{center}
\Bf{\Large Part 2: 2nd order differential equations with constant coefficients}
\end{center}

\begin{itemize}
\item Solving problems of the form \eqd{ay'' + by' + cy = 0},
 where
\[
D := b^2 - 4ac 
\begin{cases}
> 0\\
=0 \\
< 0
\end{cases}
\hspace{5cm}
\]
\item Initial and boundary value problems,
\item Problems where the right-hand side is a \Emph{polynomial},
  \Emph{exponential} or \Emph{trig function}, or the sum or product of
  these.
\item  \Bf{Power series solutions}.
\end{itemize}


}

\frame{
\begin{center}
\Bf{\Large Part 3: Integrals}
\end{center}

\begin{itemize}
\item Evaluating indefinite and definite integrals.

\item Fundamental theorem of calculus


\item Techniques of integration: Substitutions, \Emph{Integration by
    parts}, Reduction formulae; partial fractions.

\item Improper integrals: type 1 and 2, including proving that 
\begin{itemize}
\item \eqd{\int_1^\infty \frac{1}{x^p} dx \begin{cases}
\text{ diverges} & \text{ for } p\leq 1\\
\text{ converges} & \text{ for } p>1
\end{cases}
}
\item \eqd{\int_0^1 \frac{1}{x^p} dx \begin{cases}
\text{ converges} & \text{ for } p<1\\
\text{ diverges} & \text{ for } p\geq 1
\end{cases}
}

\item The comparison test.
\end{itemize}

\end{itemize}
}


\frame{
\begin{center}
\Bf{\Large Part 4: First Order Differential Equations}
\end{center}

\begin{itemize}
\item Separable, 

\item Homogeneous, 

\item Linear ($\rightarrow$ Integrating factors)

\end{itemize}
}





\end{document}


