% MA211 - Lecture 11
\documentclass[pdftex, xcolor=pdftex, dvipsnames,handout]{beamer}

\usetheme{MA211}
\usepackage{thumbpdf}
\usepackage{wasysym}
%\usepackage{ucs}
\usepackage[utf8]{inputenc}
\usepackage{pgf,pgfarrows,pgfnodes,pgfautomata,pgfheaps,pgfshade}
\usepackage{verbatim}

\usepackage{eurosym}
\usepackage{euler}

\usepackage{calc}               % Simple computations with LaTeX variables
%\usepackage[hang]{caption2}     % Improved captions

\usepackage{graphicx}           % Standard graphics package

\usepackage{amsmath, amsthm, amssymb}


\newcommand{\fquad}{\mbox{\qquad}}
\newcommand{\bull}{$\bullet$ }

\newcommand {\I} {\mathcal I}
\newcommand {\calI} {\mathcal I}
\def\disint{\displaystyle\int}

\DeclareMathOperator{\D}{d}
\newcommand{\dydx}{\frac{\D y}{\D x}}

%\definecolor{gray}{rgb}{0.69, 0.69, 0.69} \newcommand{\gray}[1]{\textcolor{gray}{#1}}
\definecolor{dogreen}{rgb}{0.33, 0.42, 0.18} \newcommand{\dogreen}[1]{\textcolor{dogreen}{#1}}
\definecolor{maroon}{rgb}{.5,0.2,0.2}\newcommand{\maroon}[1]{\textcolor{maroon}{#1}}
\definecolor{greena}{rgb}{.1,0.581,0.1}\newcommand{\greena}[1]{\textcolor{greena}{#1}}

\definecolor{blue4}{rgb}{0,0,.545}
\newcommand{\Blue}[1]{\textcolor{blue}{#1}}
\newcommand{\Red}[1]{\textcolor{red}{#1}}
\definecolor{pink}{rgb}{1.,0.75,0.8}
\definecolor{darkred}{rgb}{0.5,0.0,0.0}
\definecolor{darkgreen}{rgb}{0,0.3,0.3}
\definecolor{purple}{rgb}{0,0.3,0.3}
\definecolor{darkblue}{rgb}{0.0, 0.0, .5}
\definecolor{dpurple}{rgb}{.3,.0,.3}
\newcommand{\Green}[1]{\textcolor{darkgreen}{#1}}
\newcommand{\DRed}[1]{\textcolor{darkred}{#1}}
\newcommand{\DBlue}[1]{\textcolor{darkblue}{#1}}
\newcommand{\Purple}[1]{\textcolor{dpurple}{#1}}
\newcommand{\Emph}[1]{\textcolor{darkred}{\textbf{\it #1}}}
\newcommand{\remph}[1]{\textcolor{darkred}{\textbf{\emph{#1}}}}
\newcommand{\bemph}[1]{\textcolor{darkblue}{\textbf{\emph{#1}}}}
\newcommand{\gemph}[1]{\textcolor{darkgreen}{\textbf{\emph{#1}}}}
\newcommand{\Bf}[1]{\textcolor{darkblue}{\textbf{#1}}}
\newcommand{\Gf}[1]{\textcolor{darkgreen}{\textbf{#1}}}
\newcommand{\Rf}[1]{\textcolor{red}{\textbf{#1}}}
\newcommand{\Rmf}[1]{\textcolor{red}{\mathbf{#1}}}

\newcommand{\Conj}[1]{\overline{#1}}

\newcommand{\code}[1]{\textcolor{darkblue}{\texttt{\textbf{#1}}}}
\newcommand{\icode}[1]{{\blue\texttt{\textbf{\emph{#1}}}}}
\newcommand{\gcode}[1]{{\Green{\texttt{\textbf{\emph{#1}}}}}}
\newcommand{\out}[1]{\texttt{\emph{\textbf{\Green{#1}}}}}





\newenvironment{vminipage}%
{\begin{Sbox}\begin{minipage}\begin{small}\begin{verbatim}}%
{\end{verbatim}\end{small}\end{minipage}\end{Sbox}\fbox{\TheSbox}}

\newenvironment{nminipage}%
{\begin{Sbox}\begin{minipage}}%
{\end{minipage}\end{Sbox}\fbox{\TheSbox}}


\let\Arg\relax\DeclareMathOperator{\Arg}{\mathtt{Arg}}
\let\Arg\relax\DeclareMathOperator{\e}{\mathtt{e}}

\newcommand {\AND} {\wedge}
\newcommand {\OR} {\vee}
\newcommand {\NOT} {\neg}
\newcommand {\IMPLIES} {\rightarrow}
%\newcommand {\IFF} {\leftrightarrow}
\renewcommand {\iff} {\Leftrightarrow}
\newcommand {\NAND} {\uparrow}
\newcommand {\NOR} {\downarrow}
\newcommand {\XOR} {\otimes}

\newenvironment{citemize}% Colour items
{\begin{description}}%
{\end{description}}

\newcommand {\maroonitem}{\item[\maroon{$\bullet$}]}

\newcommand {\gitem} {\item {\includegraphics[width=.4cm,angle=-10]{img/green-bullet-on-white.ps}}}
\newcommand {\ritem} {\item {\includegraphics[width=.4cm,angle=-10]{img/red-bullet-on-white.ps}}}
\newcommand {\yitem} {\item {\includegraphics[width=.4cm,angle=-10]{img/yellow-bullet-on-white.ps}}}
\newcommand {\bitem} {\item {\includegraphics[width=.4cm,angle=-10]{img/blue-bullet-on-white.ps}}}

\newcommand {\greenitem} {\item {\includegraphics[width=.4cm,angle=-10]{img/green-bullet-on-white.ps}}}
\newcommand {\reditem} {\item {\includegraphics[width=.4cm,angle=-10]{img/red-bullet-on-white.ps}}}
\newcommand {\yellowitem} {\item {\includegraphics[width=.4cm,angle=-10]{img/yellow-bullet-on-white.ps}}}
\newcommand {\blueitem} {\item {\includegraphics[width=.4cm,angle=-10]{img/blue-bullet-on-white.ps}}}

\newcommand {\eq}[1]%
  {$\DBlue{#1}$}
\newcommand {\eqd}[1]%
  {$\displaystyle\DBlue{#1}$}
%\newcommand{\eq}[1]{\boldmath \DBlue{$#1$}}


\newcommand {\csf}{\centerslidesfalse}
\newcommand {\cst}{\centerslidestrue}

\newcommand {\vecii}[2] {   \big(\begin{smallmatrix} #1 \\ #2 \end{smallmatrix}\big)}
\newcommand{\atwo}[2]{\left(\!\!\begin{array}{c} #1 \\ #2 \end{array}\!\!\right)}


\newcommand{\C}{\mathbb{C}}
\newcommand{\Q}{\mathbb{Q}}
\newcommand{\R}{\mathbb{R}}
\newcommand{\N}{\mathbb{N}}
\newcommand{\Z}{\protect\mathbb{Z}}  % protect for index.
\newcommand {\Rs}{ \mathbb{R}}
\newcommand {\Cs}{ \mathbb{C}}
\newcommand {\Rnn}{ \mathbb{R}^{n \times n}}
\newcommand {\Rn}{ \mathbb{R}^{n}}


\newcommand{\mblock}{%
\setbeamercolor*{block title}{bg=maroon,fg=white}
\setbeamercolor*{block body}{bg=white,fg=maroon}
}%

\newcommand{\bblock}{%
\setbeamercolor*{block title}{bg=Steel,fg=white}
\setbeamercolor*{block body}{bg=Mylightgray,fg=Steel}
}%

\newcommand{\gblock}{%
\setbeamercolor*{block title}{bg=Green,fg=white}
\setbeamercolor*{block body}{bg=Mylightgray,fg=darkgreen}
}%


\newcommand{\rblock}{%
\setbeamercolor*{block title}{bg=Red,fg=white}
\setbeamercolor*{block body}{bg=white,fg=Black}
}%


\newcommand{\TakeNotes}{\includegraphics[width=2cm]{TakeNote}}

\def\eps{\varepsilon}
\newcommand {\del}[2]{ {\frac{\partial #1}{\partial #2}}}
\newcommand {\x}[1]{x^{[#1]}}
\newcommand {\delx}{ {\frac{\partial}{\partial x}}}
\newcommand {\delt}{ {\frac{\partial}{\partial t}}}
\newcommand {\dely}{ {\frac{\partial}{\partial y}}}
\newcommand {\ith}{{(i)}}
\renewcommand {\vec}[1]{ {\boldsymbol{#1}}}
\newcommand {\Oh} {\mathcal O}
\newcommand {\Err} {\mathcal E}
%\newcommand {\th} {\mathrm{th}}
\DeclareMathOperator{\fl}{fl}
\DeclareMathOperator{\sign}{sign}
\DeclareMathOperator{\Cond}{Cond} 
\DeclareMathOperator{\cond}{cond}
\DeclareMathOperator{\diag}{diag} 
\DeclareMathOperator{\sym}{sym} 
\DeclareMathOperator{\Trace}{Trace}
%\DeclareMathOperator{\D}{d}
\DeclareMathOperator{\E}{e}

\newcommand {\Rsym}{{ \mathbb{R}^{n \times n}_\mathrm{sym}}}

\newcommand {\st} {\mathrm{st}}
\newcommand {\nd} {\mathrm{nd}}


\parskip .25cm


\theoremstyle{definition}
\newtheorem{exercise}{Exercise}[section]
\newtheorem{method}{Method}[section]

\newcommand{\Header}[1]{\begin{center}{\Large \Bf{#1}}\end{center}}

\subtitle{MA211}
\title{Lecture 13: Nonhomogeneous DEs}

\author{Dr Niall Madden}

\date{\Large Monday, $20^\mathrm{th}$ October  2008}


\begin{document}

\setcounter{framenumber}{-1}
\frame{

\begin{block}{}
\begin{center}
{\large \insertsubtitle}

\vspace{.1cm}

\begin{large}
\textbf{\inserttitle}
\end{large}

\vspace{.15cm}

% {\footnotesize \insertauthor}

\vspace{.3cm}

{ {\insertdate}}
\end{center}
\end{block}


\vspace{-0.25cm}
\begin{center}
%\includegraphics[height=4cm]{images/SHM.jpg}

\end{center}
}




\frame{
  \frametitle{This morning's class:}
\begin{columns}[c]
\column{0.5\textwidth}
 \tableofcontents
\column{0.5\textwidth}
For further details and examples, look at the section on
\Emph{Nonhomogeneous Linear Equations}, Section 17.2 of Stewart
\Emph{Calculus: early transcendentals}.

\end{columns}
}







\section{Non-homogeneous Problems}

\frame{
So far this week, we  have looked at differential equations that have
a zero right-hand side:
\bblock
\begin{block}{Homogeneous}
\[ ay'' + by' + cy = \Rf{0}. \]
\end{block}
\pause
Now we'll look at non-homogeneous problems:
\vspace{-0.2cm}
\rblock
\begin{block}{Non-Homogeneous}
\[ ay'' + by' + cy = \Rf{f(x)}. \]
\end{block}
\pause
\vspace{-0.3cm}
The key idea will be to 
\vspace{-0.3cm}
\begin{itemize}
\item First compute the general solution to the complimentary \alert{Homogeneous} problem
\item Work out what we need to add to  this to get the solution to the
\alert{Non-homogeneous Problems}
\end{itemize}
}

\frame{
\rblock
\begin{block}{Non-Homogeneous}
\[ ay'' + by' + cy = \Rf{f(x)}. \]
\end{block}
\pause

The cases we'll consider are 
\begin{enumerate}[<+->]
\item \eq{f} is a polynomial.
\item \eq{f=Me^{Tx}} where \eq{M} and \eq{T} are constant.

\item \eq{f} is a trig function, such as \eq{\sin} and \eq{\cos}

\item Some combination of the above.
\end{enumerate}

The technique we shall use is sometimes called the \Emph{method of
  undetermined  coefficients}.

}

\subsection{General Technique}

\frame{

Suppose we want to solve
\eqd{ ay''(x) + by'(x) + cy(x) = \Rf{f(x)}}.


\begin{block}{Step 1}
Solve the corresponding \Emph{homogeneous} problem:
\vspace{-0.3cm}
\[ ah''(x) + bh'(x) + ch(x) = \Rf{0}.\]
\end{block}

\begin{block}{Step 2}
Chose a suitable function \eq{u} and substitute it into the DE
\vspace{-0.3cm}
\[ au''(x) + bu'(x) + cu(x) = \Rf{f}\]
%\vspace{-0.5cm}
to determine its coefficients. This is called a \Emph{particular} solution.
\end{block}
\gblock
\begin{block}{Step 3}
To get the general solution to the original problem, set
\vspace{-0.3cm}
\[ y(x) = h(x) + u(x).\]
\end{block}

}

\frame{

\begin{block}{Theorem}

If \eq{h} is the general solution to: \\ \eqd{ah''(x) + bh'(x) + ch(x) = 0}\\
and \eq{u} is a particular solution of \\
\eqd{u''(x) + bu'(x) + cu(x) = f(x)},\\
 then  
\eqd{y(x) = h(x) + u(x)} is the general solution of
\eqd{ ay''(x) + by'(x) + cy(x) = f(x)}.
\end{block}
\vspace{4cm}


}



\subsection{Examples}


\section{$f(x)$ is a polynomial}
\frame{

\rblock
\begin{block}{$f$ is a polynomial}
When solving the Non-homogeneous DE\\
\hspace{2cm} \eq{  ay'' + by' + cy = \Rf{f(x)}}.\\
where \eq{f} is a polynomial of degree \eq{n}:\\
\hspace{2cm} \eqd{f(x) = p_0 + p_1x + p_2x^2 + \dots +p_nx^n.}
\begin{enumerate}[<+->]
\item Solve the homogeneous DE \eq{ah'' + bh' + ch = \Rf{0}}.

\item Let \eq{u} a polynomial of the same degree as \eq{f}:\\
\hspace{2cm} \eqd{u(x) = q_0 + q_1x + q_2 x^2 + \cdots + q_n x^n.}

\item Substitute \eq{u} into the DE and solve (in order) for \eq{q_n},
  \eq{q_{n-1}}, \dots, \eq{q_1}, \eq{q_0}.
\item The general solution is then \eq{y(x)= h(x) + u(x)}.
\end{enumerate} 

\end{block}
}

\frame{
\begin{example}[$f \equiv 1$]
Find the general solution to the non-homogeneous problem:
\[
y'' + y' - 2y = 1.
\]
\end{example}

\vspace{4cm}
}


\frame{
\begin{example}[$f(x) = x+2$]
Find the general solution to the non-homogeneous problem:
\[
y'' + y' - 2y = x+2.
\]
\end{example}

\vspace{4cm}
}
\frame{
\begin{example}[$f(x) = x^3+1$]
Find the general solution to the non-homogeneous problem:
\[
y''  - y = x^3 +1.
\]
\end{example}

\vspace{4cm}
}

\frame{
\begin{exercise}[Q13.1]
Find general solutions to the following differential equations:
\begin{enumerate}
\item  $y'' + y' - 2y =1$.
\item $y'' - 6y' + 9y = x$.
\item $y'' - 2y' = x^2 + 4$.
\item  $y'' = 4x^3$.
\end{enumerate}
\end{exercise}

}
\section{$f(x)=Me^{Tx}$}


\frame{
If the right-hand side of the DE is an exponential function:
\rblock
\begin{block}{\eq{f = Me^{Tx}}}
When solving the Non-homogeneous DE\\
\hspace{2cm} \eq{  ay'' + by' + cy = \Rf{f(x)}}.\\
where \eq{f = Me^{Tx}}:\\
\begin{enumerate}%[<+->]
\item Solve the homogeneous DE \eq{ah'' + bh' + ch = \Rf{0}}.

\item Check if term \eq{e^Tx} appears in \eq{h}
\begin{itemize}
\item  If it doesn't,  set  \eq{u = Me^{Tx}}. 
\item If it does,  set  \eq{u = Mxe^{Tx}}, or \eq{u = Mx^2e^{Tx}}.
\end{itemize}
(\alert{More about this later})

\item Substitute \eq{u} into the DE, divide by \eq{e^{Tx}} and solve
  for \eq{M}.
\item The general solution is then \eq{y(x)= h(x) + u(x)}.
\end{enumerate} 
\end{block}
}


\frame{
\begin{example}[$f(x) = e^{2x} $]
Find the general solution to the non-homogeneous problem:
\[
y''  - y = e^{2x}.
\]
\end{example}

\vspace{4cm}
}


\frame{
\begin{example}[$f=e^{-3x}$]
Find the general solution to the non-homogeneous problem:
\[
y''  - \sqrt{7} y' + 2 y = e^{-3x}.
\]
\end{example}

\vspace{4.5cm}
}

\end{document}
\frame{
\begin{example}
Solve the following DE
\[
2 y''  +  y' - y = 3e^{x/2}.
\]
\end{example}

\vspace{4.5cm}
}

\frame{
\begin{example}
Solve the following DE
\[
 y''  +  2y' + y = 4e^{-x}.
\]
\end{example}

\vspace{4.5cm}
}



\frame{
\begin{exercise}[Q13.2]
Find general solutions to the following non-homogeneous differential equations:
\begin{enumerate}
\item $y'' + y' - 2y =e^{-x}$.
\item $y'' + y' - 2y = 3e^x$.
\item $y'' + 5y' + 6y = 4e^{-2x}$.
\item  $-3y'' + 3y' -y = \frac{1}{2}e^{-x/2}$.
\end{enumerate}

\end{exercise} 

}

\section{$f$ is a trigonometric function}

\frame{
%If the right-hand side of the DE is a trig function (\eq{\cos} or
%\eq{\sin}):
\rblock
\begin{block}{$f$ is $\sin(Tx)$ or $\cos(Tx)$}
When solving the Non-homogeneous DE\\
\hspace{2cm} \eq{  ay'' + by' + cy = \Rf{f(x)}}.\\
where \eq{f} is \eq{\sin(Tx)} or \eq{\cos(Tx)}:\\
\begin{enumerate}[<+->]
\item Solve the homogeneous DE \eq{ah'' + bh' + ch = \Rf{0}}.
\begin{itemize}
\item If \eq{f} does \Emph{not} appear as part of \eq{h}, set
 \eq{u = z_1 \sin(Tx) + z_2\cos(Tx)}, where \eq{z_1} and
  \eq{z_2} are some numbers.
\item If \eq{f}  \Emph{does} appear as part of \eq{h}, set
 \eq{u = z_1 x\sin(Tx) + z_2 x\cos(Tx)}.
\end{itemize}
\item Substitute \eq{u} into the DE. Get 2 equations: one for
  \eq{\sin}  and one for \eq{cos}.
\item Solve the 2 equations for \eq{z_1} and \eq{z_2}.

\item The general solution is then \eq{y(x)= h(x) + z_1 \sin(x) + z_2 \cos(x)}.
\end{enumerate} 
\end{block}
}

\frame{
\begin{example}
Find the general solution to the non-homogeneous problem:
\[
4 y''  + 12 y' + 9y = \sin(\frac{x}{2}).
\]

\end{example}

\vspace{4.5cm}
}

\frame{
\begin{example}
Find the general solution to the non-homogeneous problem:
\[
4 y''  + 12 y' + 9y = \sin(\frac{x}{2}).
\]

\end{example}

\vspace{4.5cm}
}



\frame{
\begin{example}
Find the general solution to the non-homogeneous problem:
\[
y''  + 4 y = \cos(2x).
\]

\end{example}

\vspace{4.5cm}
}

\frame{
\begin{exercise}[Q13.3]
 Find general solutions to the following differential equations:
\begin{enumerate}
\item $y'' - y =\cos(x)$.
\item $y'' + y' - 2y =5\sin(-2x)$.
\end{enumerate}
\end{exercise}
}
\end{document}

