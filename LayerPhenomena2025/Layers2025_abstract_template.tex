% Template for post or presentation abstracts submitted to the Laayers
% 2025 Workshop in Galway
% 1. Fill in the presenter's information (name, affiliation, email
% address) and talk information (title of presentation, abstract).
% 2. Compile to make sure there are no errors.
% 3. Save the .tex file for your abstract with a distinctive name,
% ideally "FamilynameGivenname.tex".
% 4. Send your file by email to Niall.Madden@UniversityOfGalway.ie
%   and include the word "Layers" somewhere in the subject line
%       
% * Please keep your LaTeX commands simple, and avoid the use of
%    user-defined macros.
% * Include details of co-authors and funding acknowledgements after the abstract.
% * Questions: Contact Niall Madden (Niall.Madden@UniversityOfGalway.ie)

\documentclass[a4paper]{amsart} 
\usepackage{amssymb} 
\usepackage{hyperref}
\pagestyle{empty}
\date{} 

%% IGNORE THE NEXT 4 LINES
\makeatletter % 
\def\labelenumi{\theenumi.}
\def\theenumi{\@arabic\c@enumi}
\makeatother
%% STOP IGNORING NOW

\begin{document}

\title{A sample \LaTeX\ abstract for a talk ts the Layer Phenomena
  Workshop, 2025}
\author{Niall Madden} % Presenting author only; do not include
                      % co=-authors; list those in the abstract.
\address{University of Wherever} % Affiliation of presenting author
\email{Madden@whereever.edu} % Email address of presenting author only

\maketitle

% The abstract, ideally no more than one page.

This is a sample abstract for talks and posters
presented at the \emph{21st Workshop on Numerical Methods for Problems
  with Layer Phenomena} to be held at University of Galway (Ireland),
24--25 April 2025.

Please, follow the instructions at
\url{https://www.niallmadden.ie/LayerPhenomena2025/registration.html} 
for submitting your abstract. 

Try to keep your file as simple as possible, so that it is easily
rendered in various formats. Give the file a name that readily
identifies it with you, for example \texttt{GIShishkin.tex}. 

If you wish to include citations, you can do so like this:
\cite{My-key-for-Bakhvalov69}. Use a bibitem key that is likely be be unique to
you (e.g., don't use the default Math Reviews key).
Any uncited bibliography entries will also
appear. If you have none, then delete that section.

%% Coauthor details here (optional - delete if not applicable)
\emph{This is joint work with A.N. Other (Berlin).}

% Funding acknowledgment (optional - delete if not applicable)
\emph{Supported by the National Mathematics Foundation, Grant XYZ-123.}

\begin{thebibliography}{1}

\bibitem{My-key-for-Bakhvalov69}
Bakvalov, N.: Towards optimization of methods for solving boundary value
  problems in the presence of boundary layers.
\newblock Zh. Vzchisl. Mat. i Mat fiz. \textbf{9}, 841--859 (1969)

\bibitem{my-key-for-ClGr05a}
Clavero, C., Gracia, J., O'Riordan, E.: {A parameter robust numerical method
  for a two dimensional reaction-diffusion problem.}
\newblock Math. Comput. \textbf{74}(252), 1743--1758 (2005).
\newblock \doi{10.1090/S0025-5718-05-01762-X}


\bibitem{Proceedings-example}
Gracia, J.L., Madden, N., Nhan, T.A.: Applying a patched mesh method to
  efficiently solve a singularly perturbed reaction-diffusion problem.
\newblock In: H.G. Bock, H.X. Phu, R.~Rannacher, J.P. Schl{\"o}der (eds.)
  Modeling, Simulation and Optimization of Complex Processes HPSC 2015, pp.
  41--53. Springer International Publishing, Cham (2017)

\end{thebibliography}


\end{document}


% Please leave the next bit alone  
%%% Local Variables: 
%%% mode: latex
%%% TeX-master: "Layers2025.tex"
%%% End:
