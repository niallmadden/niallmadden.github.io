\documentclass[a4paper, 10pt]{article}
 
\usepackage[body={18.5cm,26cm},centering, nohead, nofoot]{geometry}

\usepackage{amsmath,amsthm, amssymb}
\usepackage{graphics, graphicx, pictex}

%\input{../Lectures/CS319stuff.tex}

\newcommand {\eq}[1]{$\DBlue{#1}$}
\newcommand {\eqd}[1]{$\displaystyle \DBlue{#1}$}


\usepackage{enumerate}
\usepackage{hyperref}

\usepackage{inconsolata}

\usepackage{listings}
\lstset{language=C++}
\usepackage{xcolor}
\definecolor{BG}{rgb}{.95,.95,1}
\definecolor{darkblue}{rgb}{0.0, 0.0, .78}
\newcommand{\DBlue}[1]{\textcolor{darkblue}{#1}}
\newcommand{\code}[1]{{\textcolor{blue}{\texttt{\textbf{#1}}}}}

\newcommand{\icode}[1]{{\textcolor{blue}{\texttt{\textbf{\emph{#1}}}}}}
\newcommand{\rcode}[1]{{\textcolor{red}{\texttt{\textbf{#1}}}}}
\newcommand{\lcode}[1]{\code{\large #1}}


\newcommand{\cnset}[1]{\lstset{language=C++,
    showstringspaces=false, 
    basicstyle=\ttfamily\footnotesize,
    keywordstyle=\color{blue}\footnotesize,
    commentstyle=\color{black!60}\rmfamily\scriptsize,
    frame=single,
    backgroundcolor=\color{BG},
    texcl=true,
    linewidth=#1\textwidth,
    numbers=left,
    numberstyle=\footnotesize,  stepnumber=2,
    numbersep=5pt,
    numberblanklines=false}%
}

%\newcommand {\newsec}{%
%\begin{center}
%  \rule[3pt]{3cm}{0.5pt} O \rule[3pt]{3cm}{0.5pt} 
%\end{center}%
%}
\newcommand {\newsec}{\dotfill}


\renewcommand {\sectionautorefname}{Section}
\newcommand {\propositionautorefname}{Proposition}
\newcommand {\propautorefname}{Proposition}
\renewcommand {\subsectionautorefname}{Section}
\newcommand {\lemmaautorefname}{Lemma}

%% Fancy block/box stuff
\usepackage{fancybox}
\newenvironment{Obox}%
{\begin{Sbox}\begin{minipage}{0.9\textwidth}\begin{center}\begin{minipage}}%
{\end{minipage}\end{center}\end{minipage}\end{Sbox}\Ovalbox{\TheSbox}}




\linespread{1.1}


%% START::  Stuff for displaying answers (or not)
\usepackage[skins]{tcolorbox}
\newtcbox{\myans}{enhanced,nobeforeafter,tcbox raise base,boxrule=0.4pt,top=0mm,bottom=0mm,
  right=0mm,left=4mm,arc=1pt,boxsep=2pt,before upper={\vphantom{dlg}},
  colframe=green!50!black,coltext=green!25!black,colback=green!10!white,
  overlay={\begin{tcbclipinterior}\fill[green!75!blue!50!white] (frame.south west)
    rectangle node[text=black,font=\sffamily\bfseries\tiny,rotate=90]
    {ANS} ([xshift=4mm]frame.north west);%
  \end{tcbclipinterior}}}

\newenvironment{AnswerBlock}
{\begin{tcolorbox}[colback=yellow!15]\textbf{Answer: }}{\end{tcolorbox}}

\newif\ifAnswer
\def\Ans#1{\ifAnswer \myans{#1} \else \relax \fi}
\def\AnsText#1{\ifAnswer%
  \begin{AnswerBlock}#1
  \end{AnswerBlock}\else\relax\fi}
%% END::  Stuff for displaying answers (or not)
%%% Comment this (or set to \Answerfalse) to remove answers
%\Answertrue
%%%%%%%%%%


\newcommand {\mycolourbox}[2] {\tcbox[on line, top=0pt,left=0pt,right=0pt,bottom=0pt,colframe=#1!50!black,coltext=#1!25!black,colback=#1!10!white]{#2}}



\author{Niall Madden}
\title{CS319 -- Scientific Computing}


\newcommand{\ChapterTitle}{ ~ }
\newcommand{\SectionTitle}{ ~ }
\pagestyle{empty}

\begin{document}


\twocolumn[
\begin{center}
\mycolourbox{lime}{{\Large CS319 \textbf{SAMPLE} test (Feb 2025)} \Ans{with solutions}}
\end{center}
]

\emph{\textbf{Instructions:}}
\begin{itemize}
  \item This is just a \emph{sample} of the type and range of
    questions you can expect for the class test on Friday 21 Feb.
\item The real test will also have 4 questions and you'll be expected
  to answer all of them.
  
\item The  solution of each question should be in the form of a C++
  program. For the test, you'll upload thse  ``Assignments... Class Test''  on
  Canvas.  You can upload a single file, or one file per question (as you
  prefer). Each of your files should include comments with your name, ID
  number, and email address.

\item The test will be  ``open book'': you can use your lecture notes,
    and any other resource at \\
    \url{https://www.niallmadden.ie/2425-CS319}
    
  \item You may not communicate with anyone during the test, use a
    search engine, or generative AI.

    \item Solutions to this sample test will be posted on Wednesday (19 Feb)
    
  \end{itemize}

  \dotfill

  \begin{enumerate}[{Q}1]
  \item Here is a simple ``Hello World'' C++ programme.
\cnset{.4}
\lstinputlisting{HelloWorld.cpp}

You can also download it from \\
  \href{https://www.niallmadden.ie/2425-CS319/ClassTest/HelloWorld.cpp}
  {\small \texttt{niallmadden.ie/2425-CS319/ClassTest/HelloWorld.cpp}}.

  Compile and run this program.   Modify it so that
  \begin{enumerate}
    \item\label{it:string} a variable of type \code{string} is declared;
    \item The user is prompted to enter their name;
    \item The user's input is read and stored in the  \code{string}
      declared in (a).
    \item A  message is displayed using that name. For example, if the user
      enters ``Catherine'' as their name, it should output
      ``\code{Hello Catherine}''.
    \end{enumerate}

  \emph{The goal of Q1 is  to test if you can compile and
    run a C++ program, define a \code{string} variable, and do basic
    input and output. Pay special attention to ensuring that your code
    compiles without error or warning}.

    
  \dotfill

\newpage
\item For this question, it helps to know that
  \begin{itemize}
  \item 
    \code{int a[10];}\\
    creates an array (list) of 10 integers called $a[0]$, $a[1]$, ...,
    $a[9]$.
  \item \code{x=rand()\%n};\\
    sets  \code{x} to be a random int between 0
  and $n-1$.
\end{itemize}

Write a  program that works as follows.
\begin{enumerate}
\item the program has a function with header\\
  {\small \code{int CountOccurences(int list[], int len,~int~k);}}\\
  which returns the number of times that \code{k} occurs in the array
  \code{list[]}, which is of length \code{len}.
\item The \code{main()} part of the program uses a \code{for} loop to
  create an array of
  integers of length 10, and sets the entries to be a random number
  between 0 and 10 (inclusive).

\item It then uses the \code{CountOccurences()} function to report
  which entries in the list are unique (that is, occur exactly once).
\end{enumerate}
    

    \emph{The goal of Q2 is to verify that you are competent writing
      \code{for}-loops and functions.}
\AnsText{\small 
  \cnset{1}
\lstinputlisting{Q2-sample.cpp}
}

\dotfill


\item  Write a recursive function with header\\
  \code{int MyNchooseK(int n, int k);}\\
  that 
  takes a two integer arguments, \code{n} and \code{k}, and
  returns \eqd{n \choose k}, using the following algorithm.
  \begin{itemize}
  \item If $n<k$, or either $n$ or $k$ are negative,
    then \eqd{ {n \choose k} =0}.
  \item Otherwise, if \eqd{k=0} or \eqd{k=n}, then
    \eqd{ {n \choose k} =1}.
  \item Otherwise \eqd{{n \choose k}= {n-1 \choose k-1} + {n-1 \choose k}}
  \end{itemize}
        
  In your \code{main()} function, verify that \code{MyNchooseK()}
   works by
  \begin{itemize}
  \item  Prompting the user to enter values of $n$ and $k$;
  \item reading in those values, using \code{std::cin}
  \item Outputting \eqd{ {n \choose k}} for these values.
  \end{itemize}

  Note: for example, \eqd{ {6 \choose -1}}=0,
  \eqd{ {6 \choose 1}}=6, and 
  \eqd{ {6 \choose 3}}=20.
  
  \emph{The purpose of Q3 is to verify that you can read input,
    write functions, and use \code{if} statements.}
        \AnsText{\small 
  \cnset{1}
\lstinputlisting{Q3-sample.cpp}
}

\end{enumerate}

\end{document}
